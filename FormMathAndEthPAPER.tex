\documentclass[11pt,a4paper]{article}
\usepackage[latin1]{inputenc}
\usepackage{amsmath}
\usepackage{graphicx}
\usepackage[colorlinks=true, breaklinks=true]{hyperref}

\title{Formalized Mathematics and Ethereum}
\author{Martin Lundfall}


%%%%%%%%%%%%%%%%%%%%%%%%%%%%%%%%%%%%%%%%%%%%%%%%%%%%%%%
\begin{document}
%%%%%%%%%%%%%%%%%%%%%%%%%%%%%%%%%%%%%%%%%%%%%%%%%%%%%%%

\maketitle

\begin{abstract}
The decentralized infrastructure of the blockchain technology Ethereum opens a new set of possibilities for sharing content in a programmable environment. In particular, it is highly suitable for creating a standard library of formalized mathematics to which everyone can add contributions. The database can in theory be automatically checked for validity and to avoid unnecessary repetitions, scanned for relevant results, or even serve as an interactive foundation of a formalized structure of knowledge. The trustless nature of Ethereum allows communities to make democratic decisions on what contributions should be included in the database, or place bounties on particularly sought functions or proofs.
\end{abstract}

\tableofcontents

\section{Introduction}\label{Formal and informal mathematics}
The term 'Formalized mathematics' might strike readers unfamiliar with the topic as a pleonasm. What could possibly be more formal than the rigor in which mathematical proofs are presented? But as we will see, a formal approach to mathematics looks fundamentally different from traditional mathematical reasoning. To illustrate what is meant by formal mathematics, consider a simple informal proof that the sum of two odd numbers is equal:

Any odd integer can be written as $2n+1, for some integer n.$
Any even number can be written as $2m, for some integer m.$
$(2n_1+1) + (2n_2+1) = 2n_1 + 2n_2 + 2 = 2(n_1 + n_2 + 1)$
The right hand side of this eqution is on the form $2m$ and therefore even.

Although this proof is blatantly trivial, whether we are satisfied with an argument of this sort or not does not only depend on its validity, but also on whether we accept the assumptions on which the reasoning is based.
For example, in order to be convinced that $2n_1+2n_2+2 = 2(n_1+n_2+1)$, we must are assumed to know the definition of addition of multiplication and the distributive property of multiplication over addition. To accept that the left hand side of the equation indeed equals the right hand side we must recognize the transitivity of the equality relation. 
Furthermore, we are assumed to know how integers are defined, and that $n_1+n_2+1$ truly is an integer without proof. It is also worth noticing that the because the offered proof does not state that all integers are either odd or equal, we can not on this proof alone rule out the possibility that the resulting sum is not both odd and even. Just the fact that our proof is written in plain english could also be a source of ambiguity.

Of course, the fact that these assumptions are omitted in the proof does not jeopoardize its vailidity. The implicit assumptions of our example proof are trivially proven. If mathematics always were to be conducted with absolute rigor and formality, it would unlikely produce any interesting results. It is widely accepted that a given mathematical proof will assume some prior knowledge and reasoning ability of the reader.

Formalized mathematics is mathematical reasoning following a set of strict rules of syntax. It requires all the steps from axioms to conclusion to be presented, and is therefore very amenable for algorithmic proof checking capable of being done by a computer. How these rules of syntax are to be defined , and in the  are many interesting philosophical questions which arises from such an endeavor.
Att använda \LaTeX\ på Linux-datorerna i datorsalen är inte svårt. Denna mall är tänkt att användas av dig så att du kan börja skriva labbrapporter och exjobbsrapporter i \LaTeX. En längre och mycket användbar manual finns att hämta här:\\
\url{http://www.ctan.org/tex-archive/info/lshort/english/lshort.pdf} 

%%%%%%%%%%%%%%%%%%%%%%%%%%%%%%%%%%%%%%%%%%%%%%%%%%%%%%%
% \texttt, \textit (kursiverat) och \textbf (fetstil) 
% är några kommandon för att ändra textens utseende
%%%%%%%%%%%%%%%%%%%%%%%%%%%%%%%%%%%%%%%%%%%%%%%%%%%%%%%


När man använder \LaTeX för att skriva rapporter o.dyl. så skriver man, liksom i \texttt{html}, text och kod i en separat fil som sedan omvandlas av ett program till en mer läsbar fil. För \texttt{html} gör en webbläsare som Firefox, Safari eller Internet explorer jobbet och visar upp resultatet. \LaTeX\ omvandlar den kod du har skrivit till en så kallad \texttt{.dvi}-fil, eller en vanlig \texttt{pdf}.

Det är tänkt att du skall kunna göra din egen rapport helt enkelt genom att ändra i den kod som ligger till grund för detta dokument du nu läser. Hämta filen \texttt{exjobbsmall.tex} (och bilderna \texttt{plus\_och\_kryss.eps} och  \texttt{plus\_och\_kryss.eps}) från \url{http://www.math.su.se/pub/jsp/polopoly.jsp?d=7941}, och spara den på lämpligt ställe.



\section{Emacs}\label{om_text}

Du kan använda vilken textredigerare som helst, men denna introduktion beskriver hur du kan använda Emacs (och att du använder Linux). Öppna en konsol, och ta dig med \texttt{cd} till den katalog där du sparade \texttt{exjobbsmall.tex}. Öppna filen genom att skriva
\begin{verbatim}
emacs exjobbsmall.tex &
\end{verbatim}
(och tryck enter). Om inte texten är färglagd, så välj i menyraden Options $\to$ Syntax Highlighting.\\
%%%%%%%%%%%%%%%%%%%%%%%%%%%%%%%%%%%%%%%%%%%%%%%%%%%%%%%
% Man kan tvinga fram en ny rad med \\
% Om LaTeX inte avstavar ett ord (bra) så kan man föreslå
%  ställen så här latex\-manu\-al\-en i den löpande texten
% Det behövs sällan, men om du vill tvinga fram ett 
%  sidbyte kan du göra det med \newpage
%%%%%%%%%%%%%%%%%%%%%%%%%%%%%%%%%%%%%%%%%%%%%%%%%%%%%%

Spara din egen kopia av mallen genom att i menyn välja File $\to$ Save Buffer As\dots, titta på listen längst ner i fönstret; där skriver du in ett lämpligt namn som måste sluta på .tex, kanske \texttt{en-fil.tex} (och trycker enter). Ändra t.ex. h�Ã�r, spara genom File $\to$ Save (current buffer), och \textit{kompilera} din kod med \LaTeX\ genom att välja TeX $\to$ TeX file.\footnote{Bredvid menyalternativen står kortkommandon, du kan t.ex. spara genom att trycka Ctrl-x och sedan Ctrl-s. När det står M betyder det Alt-tangenten, t.ex. kopiera markerad text: Alt-w. Ett annat bra kommando är Ctrl-c som avbryter ett pågående kommando.}



\section{Titta på resultatet}\label{dvi_och_pdf}

Du kan kolla på din \texttt{dvi}-fil genom att skriva
\begin{verbatim}
xdvi en-fil.dvi &
\end{verbatim}
(om du nu kallade din fil för \texttt{en-fil}). Om du vill göra om \texttt{dvi}n till en \texttt{pdf}-fil så kan du i konsolen skriva
\begin{verbatim}
dvipdf en-fil
\end{verbatim}
Denna \texttt{pdf} kan du öppna med t.ex. \texttt{xpdf} med kommandot
\begin{verbatim}
xpdf en-fil.pdf
\end{verbatim}
\texttt{xdvi} laddar om filen så fort du klickar på fönstret, men i \texttt{xpdf} kan du behöva ladda om filen efter att du ha kompilerat den. Detta gör du genom att trycka \texttt{r}.

\subsection{Pdf direkt}
Ett alternativ som rekommenderas är att hoppa över \texttt{dvi}-steget och göra en \texttt{pdf}-fil direkt, genom att kompilera genom att skriva
\begin{verbatim}
pdflatex en-fil.tex
\end{verbatim}
i konsolen istället för att göra det via Emacs.\footnote{pdf\LaTeX är förvalet under MacOSX och kanske även under Windows. Det har även den fördelen att hyperlänkar fungerar bättre.}



\section{Listor och tabeller}


Det kan vara praktiskt att punkta upp olika saker
\begin{itemize}
\item För att det blir mer lättläst.
\item För att det är roligt
\end{itemize}
Man kan också vilja göra numrerade listor
\begin{enumerate}
\item För att saker har en naturlig ordning.
\item För att det är roligt
\item Särkilt när man kombinerar
\begin{itemize}
\item vanliga listor och
\item numrerade.
\end{itemize}
\end{enumerate}
Tabeller finns det i många olika varianter, här är några exempel (titta i koden för kommentarer!):

% ''tabular'' ligger exakt där man lägger den
\begin{tabular}{c|r|r}
% kolumnerna är från vänster till höger, _c_entrerad, _h_ögerjusterad och  _h_ögerjusterad,
% och här avgränsade med vertikala linjer indikerat av |
& krona & klave \\
\hline % en horisontell linje
vinst & 12 & 34 \\
förlust & 56 & 0 
\end{tabular}

\bigskip % lite blankrader

\begin{tabular}{l @{} r @{.} l | r @{.} l}
% illustrerar hur man kan centrera på decimalpunkter med @{.}, och osynliga avgränsare: @{}
\hline
en lång text & 15 & 34 & -5 & 09 \\
\TeX & 153 & 4 & -50 & 9 \\
\hline
\end{tabular}


\subsection{Men jag har ju gjort en fin tabell i Excel\dots}

\dots och vill inte behöva skriva om den med en massa \texttt{\&}-tecken?

Lösningen är att spara ditt kalkylblad som en kommaseparerad fil med filändelse \texttt{.csv}. Öppna den i Emacs.\footnote{Man kan öppna en ny fil i samma fönster men i en annan s.k. ''buffer'' genom att välja File $\to$ Open File\dots och skriva in filens namn längst ner i fönstret. Du kan byta tillbaka till din tidigare öppna fil under menyn Buffers.}
Välj Edit $\to$ Search $\to$ Replace\dots, skriv in \texttt{,} tryck enter, skriv in \texttt{\ \&\ } (med mellanrum omkring) och tryck enter. Emacs kommer nu för varje komma fråga om du vill byta ut det mot \texttt{\ \&\ }. Tryck \texttt{y} om det passar sig, annars \texttt{n}.

Ett annat sätt är att från konsolen skriva:
\begin{verbatim}
sed 's/,/ \& /g' tabell.csv > tabell.tex
\end{verbatim}
Öppna \texttt{tabell.tex} i Emacs och klistra in i ditt dokument.



\section{Matematiska formler och uttryck}

Den stora vinsten med att använda \LaTeX\ är att det blir lätt att skriva matematiska uttryck. Uttryck som skall stå inne i en text börjar med \$, så här $P(X=k)=\binom{13}{k} (\frac{1}{3})^k (\frac{2}{3})^{13-k}$. Men kan också låta uttrycket få större utrymmme genom att använda \$\$:
$$
P(X=k)=\binom{13}{k} \left( \frac{1}{3} \right)^k \left( \frac{2}{3}\right)^{13-k}. % Här använder jag \left och \right för att matchande parenteser skall få rätt storlek.
$$
eller \texttt{equation*}:
\begin{equation*}
P(X_3 \leq 4) = P(X_3 \geq 5) = \sum_{k=5}^{13}\binom{13}{k} \left( \frac{1}{3} \right)^k \left( \frac{2}{3}\right)^{13-k}
\end{equation*}
För flera ekvationer används lämpligen \texttt{align*}, där \texttt{\&} markerar vad ekvationerna ska ordnas efter:
\begin{align*}
P(Y \leq y) &= P\left(\frac{Y-\mu}{\sigma} \leq \frac{y-\mu}{\sigma} \right) \\
&= P\left(Z \leq \frac{y-\mu}{\sigma} \right) \\
&= \int_{-\infty}^{ (y-\mu)/\sigma } \frac{1}{ \sqrt{2\pi} } e^{ -\frac{z^2}{2} } dz \\
&= \int_{-\infty}^{ (y-\mu)/\sigma }\phi(z)dz \\
\{ \text{per def.} \}&= \Phi\left(\frac{y-\mu}{\sigma}\right)
\end{align*}
Vi använder \texttt{equation} och \texttt{align} med \texttt{*} eftersom vi inte vill ha någon \textbf{ekvationsnumrering}. Man bör bara numrera ekvationer om man refererar till dem, och då behöver de också en \texttt{label}:
\begin{equation}\label{den_haer_aer_viktig}
\text{Bin}(n,p) \approx \text{Po}(np)\quad\text{då $p$ är litet}
\end{equation}
% \quad ger lite tomt utrymme i formler. Du kan också prova \! \, \: \; \qquad
Ett annat exempel:
\begin{equation}\label{ganska_basic}
1+1=2
\end{equation}
På kursen Sanno I lär man sig att uttryck \eqref{den_haer_aer_viktig} är sant, att ekvationen \eqref{ganska_basic} är sann vet du sedan tidigare. Poängen med att använda \texttt{label}s och \texttt{eqref} är att du slipper hålla reda på de faktiska numrena. Prova att klippa ut koden till den andra ekvationen och lägga den före den första så att det ser ut så här: (kolla hur jag använder \texttt{verbatim} för längre kod-exempel!)

\begin{verbatim}
\begin{equation}\label{ganska_basic}
1+1=2
\end{equation}
\begin{equation}\label{den_haer_aer_viktig}
\text{Bin}(n,p) \approx \text{Po}(np)\quad\text{då $p$ är litet}
\end{equation}
\end{verbatim}
Kompilera (kan behöva göras två gånger!), och kolla att resultatet stämmer.


Matriser behöver man ibland:
$$
\begin{pmatrix}
a & b & c \\
d & e & f
\end{pmatrix}
\neq 
\begin{pmatrix}
12 & 3 & \rho \\
\Pi & \frac{1}{2} & \int f(x)dx \\
\end{pmatrix}
$$


\subsection{Felsökning}

Om \LaTeX\ spottar ur sig en massa varningar och felmeddelanden i nedre halvan av Emacs-fönstret beror det oftast på att
\begin{enumerate}
\item du har skrivit en \{ eller en \} för mycket eller för litet, eller så
\item har du skrivit ett eller två \$ för mycket eller för litet. 
\end{enumerate}
 Om du läser felmeddelandena så brukar det stå vilken rad felet började på (eller åtminstone var \LaTeX\ först upptäckte att något var fuffens). Använd \texttt{\%} på olika ställen på den raden eller i närheten för att kommentera bort kod som det kanske är något fel med, så att du kan se var felet finns eller inte finns.


\section{Infoga bilder och plottar}\label{infoga_bilder}

Bilder är lätt att få med. Om man använder först gör \texttt{dvi}-filer så måste bilderna vara sparade som \texttt{eps}. Om man gör \texttt{pdf}-filer direkt så måste bilderna vara sparade som just \texttt{pdf}. Du kan enkelt konvertera dina \texttt{eps}-filer till \texttt{pdf}-filer med kommandot \texttt{epstopdf}:
\begin{verbatim}
epstopdf plus_och_kryss.eps
\end{verbatim}

Det kan vara lite svårt att få bilder att ligga exakt var man vill, eftersom \LaTeX\ har en egen uppfattning om vad som ser bäst ut. Kolla var figur \ref{en_figur} hamnade, och hur jag refererade till den med \texttt{ref}

\begin{figure}[t] % Denna försöker jag lägga på _t_oppen av en sida
\centering
\includegraphics[width=9cm]{plus_och_kryss} % width kan du anpassa för att få en större eller mindre bild
\caption{Detta är två fördelningar}\label{en_figur}
\end{figure}


\section{Att referera, citera, och mer att läsa}

Man kan referera till vad som helst du har gett en \texttt{label} med kommandot \texttt{ref}. \texttt{eqref} lägger dessutom till parenteser kring (ekvations-)numret. Man kan t.ex. hänvisa till stycke \ref{om_text} som handlar om Emacs. Om din \texttt{pdf}- eller \texttt{dvi}-läsare klarar av hyperlänkar så är dessutom dina referser klickbara eftersom vi använder packaget \texttt{hyperref}.\footnote{Prova att ta bort \texttt{colorlinks=true} på rad 6 innan du skall skriva ut.}

Om du vill citera en källa gör du det enklast med \texttt{cite} så här:  Jag har lärt mig mycket om \LaTeX\ från \cite{lshort}. Om man vill veta mer om detaljer för att typsätta matematik bör man även titta på \cite{amsmath}.



\section{\LaTeX\ hemma}

Gå till \url{http://www.ctan.org/starter.html} och läs på, eller fråga din lärare!\footnote{Mer specifikt så kan du kolla in \url{http://tug.org/mactex/} för Mac, \url{http://www.tug.org/protext/} för Windows och \url{http://tug.org/texlive/} för Linux.}

\begin{thebibliography}{99} % 99 för att vi garderar oss med att vi inte kommer at ha fler än 99 ref.

\bibitem{lshort}
\textsc{Oetiker et al.} The Not So Short Introduction to \LaTeX2e,
\url{http://www.ctan.org/tex-archive/info/lshort/english/lshort.pdf}

\bibitem{amsmath}
\textsc{AMS} User's Guide for the \texttt{amsmath} Package,
\url{ftp://ftp.ams.org/pub/tex/doc/amsmath/amsldoc.pdf}

\end{thebibliography}


\end{document}   % Här slutar dokumentet


Det som står här kommer inte med. Jag brukar använda utrymmet för att spara på textsnuttar som jag inte är säker på att ta med.

\begin{figure}
\centering
\includegraphics[width=10cm]{blaha.eps}
\caption{}
\end{figure}

\overset{d}{=}

