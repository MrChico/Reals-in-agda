\documentclass[11pt,a4paper]{article}
\usepackage{amsmath}
\usepackage{graphicx}
\usepackage[colorlinks=true, breaklinks=true]{hyperref}
\usepackage{hyperref}
\usepackage{csquotes}
 % The following packages are needed because unicode
 % is translated (using the next set of packages) to
 % latex commands. You may need more packages if you
 % use more unicode characters:

 \usepackage{amssymb}
 \usepackage{bbm}
 \usepackage[greek,english]{babel}

 % This handles the translation of unicode to latex:

 \usepackage{ucs}
 \usepackage[utf8x]{inputenc}
 \usepackage{autofe}

% Some characters that are not automatically defined
% (you figure out by the latex compilation errors you get),
% and you need to define:

\DeclareUnicodeCharacter{8988}{\ensuremath{\ulcorner}}
\DeclareUnicodeCharacter{8989}{\ensuremath{\urcorner}}
\DeclareUnicodeCharacter{8803}{\ensuremath{\overline{\equiv}}}

% Add more as you need them (shouldn’t happen often).

% Using “\newenvironment” to redefine verbatim to
% be called “code” doesn’t always work properly. 
% You can more reliably use:

\usepackage{fancyvrb}

\DefineVerbatimEnvironment
  {code}{Verbatim}
  {} % Add fancy options here if you like.


\title{Formalized Mathematics and Ethereum}
\author{Martin Lundfall}


%%%%%%%%%%%%%%%%%%%%%%%%%%%%%%%%%%%%%%%%%%%%%%%%%%%%%%%
\begin{document}
%%%%%%%%%%%%%%%%%%%%%%%%%%%%%%%%%%%%%%%%%%%%%%%%%%%%%%%

\maketitle

\begin{abstract}
The decentralized infrastructure of the blockchain technology Ethereum opens a new set of possibilities for sharing content in a programmable environment. In particular, it is highly suitable for creating a standard library of formalized mathematics to which everyone can add contributions. The database can in theory be automatically checked for validity and to avoid unnecessary repetitions, scanned for relevant results, or even serve as an interactive foundation of a formalized structure of knowledge. The trustless nature of Ethereum allows communities to make democratic decisions on what contributions should be included in the database, or place bounties on particularly sought functions or proofs.
\end{abstract}
\clearpage
\tableofcontents{}
\clearpage
\section{Introduction}\label{sec: Introduction}
\subsection{Formal and informal mathematics}\label{subsec: formal & informal}
The term 'Formalized mathematics' might strike readers unfamiliar with the topic as a pleonasm. What could possibly be more formal than the rigor in which mathematical proofs are presented? But as we will see, a formal approach to mathematics looks fundamentally different from traditional mathematical reasoning. The formalized approach to mathematics strives to lay out the complete path from axioms and definitions to proofs and results, without any step along the way being subject to interpretation. To illustrate what is meant by this, consider a simple informal proof showing that the sum of two odd numbers is equal:\\

Any odd integer can be written as $2n+1$, for some integer $n$.\\
Any even number can be written as $2m$, for some integer $m$.\\
The sum of two odd numbers can be rewritten as:\\
$(2n_1+1) + (2n_2+1) = 2n_1 + 2n_2 + 2 = 2(n_1 + n_2 + 1)$\\
The right hand side of this eqution is on the form $2m$ and therefore even.\\

The proof is blatantly trivial and its result comes as no surprise. However, since it doesn't explicitly show the complete journey from axioms to conclusions, it is essentially informal. To be satisfied with a proof like this it requires us to accept the implicit assumptions on which the reasoning is based.\\
For example, in the step $2n_1+2n_2+2 = 2(n_1+n_2+1)$, we use the definition of addition of multiplication and the distributive property of multiplication over addition, altough this is not stated. To accept that the left hand side of the equation indeed equals the right hand side we must recognize the transitivity of the equality relation. We are assumed to know how integers are defined, and accept that $n_1+n_2+1$ is an integer without proof. It is also worth noticing that the because the offered proof does not state that all integers are either odd or equal, we can not on this proof alone rule out the possibility that the resulting sum is not both odd and even. Just the fact that our proof is written in plain english could also be a source of ambiguity.\\

Of course, the implicit reasoning of this proof does not jeopoardize its vailidity. They are all both trivially proven, and generally accepted. If mathematics always were to be conducted with absolute rigor and formality, it would unlikely produce any interesting results.\\

Formalized mathematics is mathematical reasoning following a set of strict rules of syntax. It requires all the steps from axioms to conclusion to be presented, and is therefore very amenable for algorithmic proof checking that a computer can do. Much of the work in formalized mathematics is therefore done in programming languages specifically designed for this purpose. \\

\subsection{Formal mathematics in Agda}\label{subsec: Agda}
In this thesis, I will be demonstrating how to formalize the constructivist definition of real numbers as described by Erret Bishop (Grundlehren Der Mathematischen Wissenschaften) in the dependently-typed programming language of Agda. Agda is a Haskell based language that essentially deals with two fundamental elements of mathematics, sets and functions. For example, here is the definition of a natural number and addition with natural numbers in Agda:\\

\begin{code}
data ℕ : Set where
  zero : ℕ
  suc  : (n : ℕ) → ℕ

_+_ : ℕ → ℕ → ℕ
zero  + n = n
suc m + n = suc (m + n)
\end{code}

To fully understand Agda, I recommend checking out of of the guides on the Agda wiki. But for starters, one important thing to realize about Agda is that proofs are written as functions, utilizing a correspondence between computer programs and formal logic discovered by Haskell Curry and William Alvin Howard in the middle of the 20th century. The Curry-Howard correspondence can be stated as: a proof is a program, the formula it proves is a type for the program. For an example, let's look at how a formal version of the  proof that the sum of two odd numbers are even would look. First, we need to define odd and even numbers in Agda.

\begin{code}
data Even : ℕ → Set where
  evenZero : Even 0
  evenSuc : {n : ℕ} → Even n → Even (suc (suc n))
 
data Odd : ℕ → Set where
  oddOne : Odd 1
  oddSuc : {n : ℕ} → Odd n → Odd (suc (suc n))
\end{code}

Now looking at the proof function, we see that the type of this function is the exact statement that we want to prove, and the definition of the function explicitly shows why the statement is true by using the definitions of the function + and the sets Even and Odd.
\begin{code}
_o+o_ : {n m : ℕ} → Odd n → Odd m → Even (n + m)
oddOne o+o oddOne = evenSuc evenZero
oddOne o+o oddSuc b = evenSuc (oddOne o+o b)
oddSuc a o+o b = evenSuc (a o+o b)
\end{code}

\section{Constructing real numbers in Agda}\label{sec: reals in agda}

This is an Agda implementation of the constructive definition of real numbers as given by Erret Bishop in Grundlehren Der Mathematischen Wissenschaften. Since the current agda standard library lacks many of the necessary functions on rationals to define real numbers, this work has been made possible by the contributions of the Github user Sabry at (\url{https://github.com/sabry/agda-stdlib/}).\\

I have been trying to follow Bishops constructions as closely as possible, but when formalizing the definitions I have been forced to make one small change. Whereas Bishop describes a sequence as a mapping of the stricly positive integers to rationals, in Agda it is much more convenient to have a sequence starting from zero. Thus where Bishop writes:
\blockquote{
A sequence of rational numbers is \textit{regular} if
\begin{center}
$| x_m - x_n | \leq m^{-1} + n^{-1} (m, n \in \Z^+)$
\end{center}
A \textit{real number} is a regular sequence of rational numbers.
}

Instead, we will write:
\begin{code}
--Constructs a real number by a giving a sequence and a proof that the sequence is regular
record ℝ : Set where
  constructor Real
  field
    f : ℕ -> ℚ
    reg : {n m : ℕ} -> (∣ (f n) ℚ.- (f m) ∣) ℚ.≤ ((+ 1) ÷suc n) ℚ.+ ((+ 1) ÷suc m)

\end{code}

I have also been trying to build this work with minimal change to the agda standard library, only making changes to the modules Rational.agda (and of course Real.agda). One significant change has been made in the definition of a rational number. In the current agda standard library, a rational number is required to be listed on its coprime form:
\begin{code}

record ℚ : Set where
  field
    numerator     : ℤ
    denominator-1 : ℕ
    isCoprime     : True (C.coprime? ∣ numerator ∣ (suc denominator-1))

  denominator : ℤ
  denominator = + suc denominator-1

  coprime : Coprime numerator denominator
  coprime = toWitness isCoprime
\end{code}

But this requirement makes proving theorems about rationals very messy, and is rarely used in informal mathematics. Instead, I have chosen to use the definition given by Bishop, which is more common: ($ℤ \times ℤ^x$). In agda, this becomes:
\begin{code}
record ℚ : Set where
  constructor _÷suc_
  field
    numerator     : ℤ
    denominator-1 : ℕ
\end{code}


\section{Ethereum and theorem bounties}


\end{document}   

But the line between the two of these elements is blurred by the fact that agda is dependently typed, meaning that a set can take arguments of other sets or functions as input

How these rules of syntax are to be defined , and in the  are many interesting philo sophical questions which arises from such an endeavor.
